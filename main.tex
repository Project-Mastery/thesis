\documentclass[twoside, 11pt]{article}

\usepackage{ipcb}

\begin{document}

\thispagestyle{empty}
\noindent\includegraphics[width=6.4cm]{est}

\vspace*{90pt}

\title{ECSMoS}

\subtitle{ECS based pedestrian mobility simulatior}

\vspace*{36pt}

\cambria

\info{Rafael Souza Cotrim}

\vspace*{24pt}

\noindent\textbf{\large{Orientadores}}

\info{Alexandre José Pereira Duro da Fonte}

\info{João Manuel Leitão Pires Caldeira}

\vfill

\notice{Dissertação/Trabalho de Projeto / Relatório de Estágio (Deixar apenas a designação aplicável e sem sublinhado e eliminar esta nota) apresentado à Escola Superior de acrescentar nome da unidade orgânica do Instituto Politécnico de Castelo Branco se aplicável acrescentar nome da instituição associada (caso contrário, eliminar esta nota) para cumprimento dos requisitos necessários à obtenção do grau de Mestre em designação do mestrado, realizada sob a orientação científica do categoria profissional do orientador Doutor nome do orientador, do Instituto Politécnico de Castelo Branco.
}

\vspace*{24pt}


\titledate{Data} 
%\blankpage
\newpage


\mainmatter

\section{Introduction}

\begin{itemize}
  \item (Context) What are pedestrian simulations? What are they used for?
  \item (Need) What are their flaws?
  \item (Task) How can this simulator address this need?
  \item (Result) TLDR of the results achieved
\end{itemize}

Pedestrian Dynamics is an area of study focused on understanding the movement of pedestrians, which often happens as crowds. Such studies can take the form of an analyzing of data collected from studying the movements of people in the real would [SRC], proposition and evaluation of methods for modeling crowd behavior [SRC], predicting movement patterns in certain spaces [SRC], among others [SRC?].

As such, pedestrian dynamics are of practical use when creating spaces meant to be utilized by people. A better understanding of how people move though a structure may aid during the design phase of a building, allowing for better planing of fire escape routes [SRC]. Similarly, not taking into account how people will behave may turn concerts or other large events into deadly crowd crushes or increase the number of trampling incidents [SRC]. Finally, even when there is little risk of loss of life, they may still be useful for incising the throughput of infrastructure such as trains stations, airports and others [SRC].

One of the best tools from this area of study comes in the form of simulation models. Computer simulations allow us to check how pedestrians will behave in a certain environment without having to spend time and resources on the construction and evaluations of scale models on the real world. This reduces costs and promotes fast iterative designs that may better align with the requirements of spaces.

There are many models for simulating the flow of pedestrians in an environment, however, designing and implementing models capable of reproducing phenomena seen on the real world is a complicated task. Pedestrian Dynamics is an inherently interdisciplinary science due to its object of study. It relies on concepts from areas as diverse as physics, engineering, psychology, computer science and sociology. Simpler models may take into account only the physical part of crowd movement and ignore all else, while others deal with the effects of having people with different ages [SRC], disabilities[SRC] or even states of mind such as calm and in panic [SRC].

When these models are implemented, it is often done on top of an exiting simulator or framework. These allow model authors to focus on the most relevant parts of their research while other tasks are handled by code already written and validated by others. Nevertheless, building a model on top of these simulators also comes with certain disadvantages. Their architecture imposes restrictions on how the model can operate, meaning that certain simulators may not be compatible with a model because it breaks one or more of the assumptions made when the simulator was being designed. [NOTE: Example? Here or later?]

[NOTE: There is plenty missing here. Maybe I need to give a bit more of a justification for why I believe this new simulator has merit]

With this in mind, I propose a new simulator for pedestrian dynamics: ECSMoS, Entity Component Systems Mobility Simulator. This simulator is based on the Entity Component Systems (ECS) architecture and has the objective of being as flexible as possible for model authors and implementers while maintaining a good performance.

[NOTE:...]

\section{Overview of the locomotion models, frameworks and architectures} \label{overview}

\subsection{Pedestrian locomotion Models}

\subsection{Existing simulation frameworks}

\begin{itemize}
  \item Vadere
  \item Menge
  \item JuPedSim
  \item SUMO
  \item FDS+Evac ?
  \item MomenTUMv2 ?
\end{itemize}

\section{ECS architecture}

\section{ECSMoS implementation}

\subsection{Chosen Technologies}

\begin{itemize}
  \item Rust
  \item Bevy ECS
\end{itemize}

\subsection{ECSMoS Architecture}
\subsection{Features?}

\section{Evaluation}
\section{Conclusion}

\end{document}