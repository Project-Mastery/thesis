\documentclass[twoside, 11pt]{article}

\usepackage{ipcb}

\begin{document}

\thispagestyle{empty}
\noindent\includegraphics[width=6.4cm]{est}

\vspace*{90pt}

\title{ECSMoS}

\subtitle{ECS based pedestrian mobility simulatior}

\vspace*{36pt}

\cambria

\info{Rafael Souza Cotrim}

\vspace*{24pt}

\noindent\textbf{\large{Orientadores}}

\info{Alexandre José Pereira Duro da Fonte}

\info{João Manuel Leitão Pires Caldeira}

\vfill

\notice{Dissertação/Trabalho de Projeto / Relatório de Estágio (Deixar apenas a designação aplicável e sem sublinhado e eliminar esta nota) apresentado à Escola Superior de acrescentar nome da unidade orgânica do Instituto Politécnico de Castelo Branco se aplicável acrescentar nome da instituição associada (caso contrário, eliminar esta nota) para cumprimento dos requisitos necessários à obtenção do grau de Mestre em designação do mestrado, realizada sob a orientação científica do categoria profissional do orientador Doutor nome do orientador, do Instituto Politécnico de Castelo Branco.
}

\vspace*{24pt}


\titledate{Data} 
%\blankpage

\mainmatter



\section{Introdução}

%{\LARGE Este projete ainda está em desenvolvimento e a documentação está incompleta!}

Está é um modelo de documento \LaTeX \space que segue as regras de formatação 
para trabalhos finais de graduação, teses de mestrado e relatórios de estágio do 
Instituto Politécnico de Castelo Branco (IPCB). 
Este projeto foi feito para seguir os padrões de 2021.
Antes de usar, verifique se as normas foram alteradas!
Os documentos contendo as normas atuais de entrega para cada uma das
escolas do IPCB estão na pasta \texttt{rules} deste projeto. 
As tabelas \ref{tab:element_styles} e \ref{tab:margins} contem um resumo das normas mais importantes.

\begin{table}[h!]
    \centering
    \begin{tabular}{|c|c|c|c|c|c|}
        \hline
        Elemento            & Fonte     & Tamanho               & Alinhamento   & Cor       & Estilo\\
        \hline
        Título trabalho     & Trebuchet   & 18 pt               & Esquerda      & Padrão    & Negrito\\
        Subtítulo trabalho  & Trebuchet   & 14 pt               & Esquerda      & Padrão    & Negrito\\
        Numeração da página & Cambria   & 11 pt                 & Centro        & [89,89,89]&\\  
        Cabeçalho par       & Trebuchet & 7 pt                  & Esquerda      & [89,89,89]&\\
        Cabeçalho impar     & Trebuchet & 7 pt                  & Direita       & [89,89,89]& Negrito\\ 
        Corpo               & Cambria   & 10–12 pt              & Justificado   & Padrão    &\\
        Títulos de capítulo & Trebuchet & 14–16 pt              & Esquerda      & Padrão    & Negrito\\
        Subtítulos          & Trebuchet & 12–14 pt\footnotemark & Esquerda      & Padrão    & Negrito\\
        Subsubtítulos       & Trebuchet & 10–12 pt\footnotemark & Esquerda      & Padrão    & Negrito\\
        Legendas (Lable)    & Trebuchet & 8–10 pt               & Centro        & Padrão    & Negrito\\
        Legendas (Corpo)    & Trebuchet & 8–10 pt               & Centro        & Padrão    &\\
        \hline
    \end{tabular}
    \caption{Tamanhos e formatação dos elementos indicados pelas normas do IPCB.}
    \label{tab:element_styles}
\end{table}

\footnotetext[1]{Sempre 2pt a menos que o tamanho do título}
\footnotetext[2]{Sempre 2pt a menos que o tamanho do subtítulo}

\begin{table}[h!]
    \centering
    \begin{tabular}{|c|c|}
        \hline
        Margem & Tamanho\\
        \hline
        Superior & 3 cm\\
        Inferior & 2 cm\\
        Interna  & 3 cm\\
        Externa  & 2,5 cm\\
        \hline
    \end{tabular}
    \caption{Tamanhos das margens}
    \label{tab:margins}
\end{table}

\begin{table}
    \centering
    \begin{tabular}{|l|c|}
        \hline
        Funcionalidade & Estado \\
        \hline
        \multicolumn{2}{|l|}{Geral}\\
        \hline
        Margens & \ding{51}\\
        Cabeçalhos & \ding{51}\\
        \hline
        \multicolumn{2}{|l|}{Formatação de texto}\\
        \hline
        Formatação de títulos & \ding{51}\\
        Formatação de texto & \ding{51}\\
        Formatação de legendas  & \ding{51}\\
        \hline
        \multicolumn{2}{|l|}{Páginas especiais}\\
        \hline
        Página de capa  & \ding{51}\\
        Página de agradecimentos & \ding{53}\\
        Página  de índice  & \ding{53}\\
        \hline
        \multicolumn{2}{|l|}{Manual de uso}\\
        \hline
        VSCode  & \ding{53}\\
        Overleaf  & \ding{53}\\
        \hline
        \multicolumn{2}{|l|}{Exemplos}\\
        \hline
        Imagens  & \ding{51}\\
        Tabelas  & \ding{53}\\
        Código  & \ding{51}\\
        \hline
    \end{tabular}
    \caption{Estado do projeto}
    \label{tab:features}
\end{table}

\subsection{Estrutura do projeto}

\dirtree{%
.1 IPCB-LaTeX.
.2 .vscode — Configuração do projeto para o VSCode.
.2 fonts — Fontes não disponíveis por padrão no \LaTeX.
.2 img — Imagens.
.2 out — Documentos de saida.
.2 rules — Regras de formatação.
.2 ipcb.sty — Documento contendo a formatação compatível com as normas do IPCB.
.2 main.tex — Arquivo principal.
}

\section{Manual de uso}

\subsection{VSCode}

\begin{enumerate}
    \item Instalar o VSCode
    \item Instalar o TeX Live
    \item Instalar extensões necessárias
    \item Instalar extensões recomendadas
\end{enumerate}

\subsection{Overleaf}

\begin{enumerate}
    \item Importar modelo
\end{enumerate}

\section{Dependências}

Se você está usando o Overleaf ou baixou a versão completa do TeX Live, 
você já deve ter todos os pacotes sendo usados. Caso contrário, garanta que os seguintes 
pacotes estejam instalados para que esse modelo funcione adequadamente:

\begin{itemize}
    \item fontspec
    \item inputenc
    \item graphicx
    \item geometry
    \item blindtext
    \item ragged2e
    \item afterpage
    \item fancyhdr
    \item xcolor
    \item titlesec
    \item indentfirst
    \item dirtree
    \item babel (e o pacote de lingas apropriado)
  \end{itemize}

\section{Exemplos}

Esta secção cotem exemplos de como usar certas funcionalidades do \LaTeX \space.
Para ver o código que está gerando esses exemplos, abra o aquivo \texttt{main.tex}.
Em caso de dúvida, vale apena procurar outros exemplos na documentação do Overleaf.

\subsection{Imagens}




Este modelo inclui configurações para automaticamente procurar imagens na pasta \texttt{img}.
Por conta disso, imagens nessa pasta podem ser acessadas usando somente seu nome. 
Abaixo seguem alguns exemplos de inserção de imagem:



\begin{figure}[ht]
    \includegraphics{est.png}
    \caption{Imagem simples}
\end{figure}

\begin{figure}[ht]
    \includegraphics[center]{est.png}
    \caption{Imagem centralizada}
\end{figure}

\begin{figure}[ht]
    \includegraphics[width=1cm,center]{est.png}
    \caption{Imagem centralizada com tamanho pré-definido (1cm)}
\end{figure}

\begin{figure}[ht]
    \includegraphics[width=\textwidth]{est.png}
    \caption{Imagem com tamanho máximo}
\end{figure}

\subsection{Tabelas}
\subsection{Código}

As normas do IPCB não especificam como código deve ser formatado no relatório. 
Aqui segue um exemplo, mas á válido ressaltar que você deve consultar seus coordenadores de projeto
ou o departamento apropriado para esclarecer essas questões. 

\begin{lstlisting}[language=Python, caption=Exemplo de uso de código]
    def add(x,y):
        return x + y
\end{lstlisting}

\subsection{Citações}

As margens do documento deverão respeitar as seguintes medidas: superior 3 cm, inferior 2 cm, interior 3 cm, exterior 2,5 cm.

A numeração deverá constar em rodapé, com um corpo de 11 alinhado ao centro. [Cor RGB 89, 89, 89] (Aplicar estilo “Numeração”) 

No cabeçalho das páginas pares deverá constar o nome do autor, em corpo 7, regular alinhado à esquerda. [Cor RGB 89, 89, 89] (Aplicar estilo “Nome do autor”)

No cabeçalho das páginas ímpares deverá constar o nome do trabalho de mestrado em corpo 7, bold, alinhado à direita. [Cor RGB 89, 89, 89] (Aplicar estilo “Nome do trabalho”)

O corpo do texto será apresentado em frente-verso, fonte “Cambria” com dimensão de 10 a 12 pontos e entrelinha de 1.15 linhas. Deverá ainda ter um alinhamento justificado e ter um avanço à primeira linha de 0,6 cm, e um espaçamento depois do parágrafo de 6 pontos. (Aplicar estilo “Normal”)

Os títulos de capítulos deverão ter um corpo de 14 a 16 pontos na fonte “Trebuchet MS” em negrito, sem avanço à primeira linha. (Aplicar estilo “Títulos”)

Todos os subtítulos deverão ter um corpo dois pontos abaixo dos títulos em negrito, sem avanço à primeira linha e constar na fonte “Trebuchet MS”. (Aplicar estilo “subtítulos”)

Os subsubtítulos deverão ter um corpo dois pontos abaixo dos subtítulos em negrito, sem avanço à primeira linha e constar na fonte “Trebuchet MS”. (Aplicar estilo “subsubtítulos”)

As legendas deverão ter um corpo de 8 a 10 pontos na fonte “Trebuchet MS”, com a informação do número da legenda em negrito e o resto do texto em regular. (Aplicar estilo “Legendas”)


\end{document}